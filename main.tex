\documentclass[oneside, 12pt]{book}
\usepackage{mystyle}
\usepackage{subfiles} %essential for modular writing and subfile compilation must be last package loaded
\graphicspath{{./images/}{images/}} %this command specifies first the path relative to main file to images folder, then the path relative to sub file i think.. very important

\usepackage{hyperref}
\begin{document}
\begin{titlepage}
	\setlength{\parskip}{12pt}
	\centering
	{\textbf{\LARGE A Federated Reinforcement Learning Approach for Autonomous Vehicle Platooning}\par}
	\vspace{1.5 cm}
	{by}\par
	\vspace{1 cm}
	{Christian Boin}\par
	\vspace{2cm}
	{A thesis\\
	presented to\\
	the University of Guelph}\par
	\vspace{2cm}
	{In partial fulfillment of requirements\\
	for the degree of\\
	Master of Applied Science,\\
	Engineering (Artificial Intelligence)}\par
	\vspace{2cm}
	{\normalsize Guelph, Ontario, Canada}\par
	\vspace{0.25cm}
	{\copyright\hspace{6pt}Christian Boin, August 2022}\par
\end{titlepage}
\pagenumbering{roman}
\chapter*{Abstract}
\noindent Christian Boin \hfill Advisor: Dr. Lei Lei


\noindent University of Guelph, 2022  \hfill Co-advisor: Dr. Simon Yang
\vspace{12pt}

Since 2016 Federated Learning (FL) has been an evolving topic of discussion in the artificial
intelligence (AI) research community.  In recent years, the successful applications of FL
in supervised learning applications led to the development and study of
Federated Reinforcement Learning (FRL).  At the time of writing this thesis,
few works exist on the topic of FRL applied to
autonomous vehicle (AV) platoons. In addition, most FRL works choose a single aggregation
method (usually weight or gradient aggregation). We explore FRL's effectiveness as a means
to improve AV platooning by designing and implementing an FRL framework atop of a custom
AV platoon environment.  The application of FRL in AV platooning is studied under two
scenarios: (1)  Inter-platoon FRL (Inter-FRL) where FRL is applied to AVs across different
platoons; (2) Intra-platoon FRL (Intra-FRL) where FRL is applied to AVs within a single
platoon. Both Inter-FRL and Intra-FRL are applied to a custom AV platooning environment using
both gradient and weight aggregation to observe the performance effects FRL can have on AV
platoons relative to an AV platooning environment trained without FRL.  It is concluded that
Intra-FRL using weight aggregation (Intra-FRLWA) provides the best performance for
controlling an AV platoon. In addition, we found that weight aggregation in FRL for
AV platooning provides increases in performance relative to gradient aggregation.
Finally, a performance analysis is conducted for Intra-FRLWA versus a platooning
environment trained without FRL for platoons of length 3, 4 and 5 vehicles. It is concluded
that Intra-FRLWA largely out-performs the platooning environment that is trained without FRL.

\chapter*{Acknowledgements}
I would like to thank my advisor Dr. Lei Lei for giving me the opportunity to perform this research.
Thank you for your support and guidance, it was a massive help throughout my graduate years.  With your guidance,
I have learned how to perform research effectively, become published, and learned more than I could have imagined on
the topics of DRL, FRL, and autonomous vehicles.

In addition, I would like to thank Dr. Simon Yang for joining my advisory committee and being
generous with time whenever I needed critical review of my work.

\tableofcontents
% \clearpage\addcontentsline{toc}{chapter}{List of Tables}
\listoftables
% \clearpage\addcontentsline{toc}{chapter}{List of Figures}
\listoffigures
% \textcolor{green}{PROOF READ NEEDED}
% \textcolor{blue}{FINAL RESULT SUMMARY NEEDED}
% \textcolor{red}{MAJOR REVISION NEEDED!}
\subfile{tex/1_Introduction}
\subfile{tex/2_Background}
\subfile{tex/3_Literature}
\subfile{tex/4_Methods}
\subfile{tex/5_Results_Discussion}
\subfile{tex/6_Conclusion}


\newpage
\bibliographystyle{IEEEtran}
\addcontentsline{toc}{section}{References}
\bibliography{references}
% \newpage
% \newpage
% \appendix

\end{document}