\newcommand{\interfrlRewWidth}{0.32}
\newcommand{\interfrlRewSpace}{0cm}

\chapter{Results and Discussion} \label{chap:results}
In this chapter, the experimental setup for applying both Inter and Intra-FRL to the
AV platooning environment is presented.  The AV platooning environment and Inter/Intra
FRL algorithms are implemented in Python 3.7 using Tensorflow 2.

\section{Experimental Setup}
The parameters specific to the AV platoon environment are summarized in Table
\ref{tab:pl_hyps}.  The time step interval is $T=0.1s$, and each training episode is
composed of 600 time steps.  Furthermore, the coefficients $a$, $b$, $c$ and $d$
given in the reward function \eqref{eqn:rewardB} are a means to define how much each
component of \eqref{eqn:rewardB} contributes to the calculation of the reward. These
coefficients may be tuned in order to determine a balance amongst each component,
leading to a better optimization during training.
The coefficients were tuned using a grid search strategy and are listed
as $a=0.4, b = c = d = 0.2$.

\begin{table}[H]
    \centering
    \caption{Parameters of the AV Platoon Environment}
    \begin{tabular}{lll} \toprule
        Parameter & Value \\ \midrule
        Time step $T$ interval & 0.1 s \\ \hline
        Number of time steps per training episode & 600 \\ \hline
        Time gap $h_i$ &              1 s \\ \hline
        Driveline dynamics coefficient $\tau$ &            0.1 s \\ \hline
        Maximum absolute control input $u_{max}$ & 2.5 $m/s^2$ \\ \hline
        Reward coefficient $a$ &            0.4 \\ \hline
        Reward coefficient $b$ &            0.2 \\ \hline
        Reward coefficient $c$ &            0.2 \\ \hline
        Reward coefficient $d$ &            0.2 \\ \bottomrule
    \end{tabular}
    \label{tab:pl_hyps}
\end{table}

Each DDPG agent consists of a replay buffer, and networks for the actor, target actor,
critic and target critic.  The actor network contains four layers: an input layer for
the state, two hidden layers with 256 and 128 nodes respectively, and an output layer.
Both hidden layers use batch normalization and the relu activation function.  The
output layer uses the tanh() activation function.  The output layer is scaled by the
high bound for the control output, in this case 2.5 $m/s^2$.  The critic network is
structured with two separate input layers for state and action.  These two layers are
concatenated together, and fed into a single hidden layer before the output layer.
The layer with the state input has 48 nodes, the relu activation function and batch
normalization.  The same is applied for the action layer, but instead with 256 nodes.
The post concatenation layer uses 304 input nodes, followed by a hidden layer
with 128 nodes, again with relu activation and batch normalization applied.  The
output of the critic uses a linear activation function.  Ornstein-Uhlenbeck noise is
applied to the model's predicted action, $u_i$. The structure of the models are
presented in Figures \ref{fig:actB} and \ref{fig:crticB}. All except the final layers
of the actor and critic networks were initialized within the range
\begin{tiny}$\left[-\dfrac{1}{\sqrt{fan\:in}},\:\dfrac{1}{\sqrt{fan\:in}}\right]$\end{tiny},
where-as the final layer is initialized using a random uniform
distribution bounded by $[-3\times10^{-3},\:3\times10^{-3}]$.
Table \ref{tab:ddpg_hyps} presents the hyperparameters specific to the DDPG algorithm.

\begin{figure*}
\centering
\begin{subfigure}{.4\textwidth}
    \centering
    \includesvg[inkscapelatex=false, width=0.75\linewidth]{assets/actor.svg}
    \caption{The actor network for $V_i$.}
    \label{fig:actB}
\end{subfigure}%
\begin{subfigure}{.6\textwidth}
    \centering
    \includesvg[inkscapelatex=false, width=1.0\linewidth]{assets/critic.svg}
    \caption{The critic network for $V_i$.}
    \label{fig:crticB}
\end{subfigure}
\caption{Actor and critic networks for $V_i$.}
\label{fig:test}
\end{figure*}



\begin{table}[H]
    \centering
    \caption{Hyperparameters for the DDPG Algorithm}
    \begin{tabular}{lll} \toprule
        Hyperparameter & Value \\
        \midrule
        Actor learning rate & 5e-05\\ \hline
        Critic learning rate & 0.0005\\ \hline
        Batch size & 64\\ \hline
        Noise & Ornstein-Uhlenbeck process with $\theta=0.15$, $\sigma=0.02$ \\ \hline
        Weights and Biases  & Random uniform distribution $[-3\times10^{-3},\:3\times10^{-3}]$ (final layer), \\
        Initialization    & \begin{tiny}$\left[-\dfrac{1}{\sqrt{fan\:in}},\:-\dfrac{1}{\sqrt{fan\:in}}\right]$\end{tiny} (other layers)\\        \bottomrule
    \end{tabular}
    \label{tab:ddpg_hyps}
\end{table}

The hyperparameters specific to Inter and Intra-FRL are presented in Table \ref{tab:frlhyps}.
During a training session with FRL, both local updates and FRL updates with aggregated
parameters are applied to each DDPG agent in the system. FRL updates usually
occur at a given frequency known as the FRL update delay, and furthermore FRL updates
may be terminated at a specific training episode as defined by the FRL cutoff ratio.  The
FRL update delay is defined as the time in seconds between FRL updates during a training
episode. The FRL cutoff ratio is the ratio of the number of episodes where FRL updates
are applied divided by the total number of episodes in a training session. Note that the
aggregation method denotes whether the model gradients or weights are averaged during
training using FRL.
\begin{table}[H]
    \caption{FRL Specific Initial Hyperparameters}
    \centering
    \begin{tabular}{llll}
        \toprule
        FRL Type & Aggregation Method & Hyperparmeter & Value \\
        \midrule
        Inter-FRL & Gradients & FRL update delay &  0.1 \\
        Inter-FRL & Gradients & FRL cutoff ratio &  0.8  \\
        Inter-FRL & Weights & FRL update delay &  30  \\
        Inter-FRL & Weights & FRL cutoff ratio &  1.0   \\
        Intra-FRL & Gradients & FRL update delay &  0.4  \\
        Intra-FRL & Gradients & FRL cutoff ratio &  0.5  \\
        Intra-FRL & Weights & FRL update delay &  0.1  \\
        Intra-FRL & Weights & FRL cutoff ratio &  1.0  \\
        \bottomrule
    \end{tabular}
    \label{tab:frlhyps}
\end{table}

For the purposes of this study, an experiment is defined as a training session for a
specific configuration of hyper-parameters, using the algorithm defined in Algorithm 1.
During each experiment training session, model parameters were trained through the base
DDPG algorithm or FRL in accordance with Algorithm 1. Once training has concluded, a
simulation is performed using a custom built evaluator API.  The evaluator performs
simulations for a single 60 second episode using the trained models, calculating the
cumulative reward of the model(s) in the experiment.  The entire project is designed and
implemented using Python3, and Tensorflow.  As previously stated, each vehicle in the
platoon is modelled using the CACC model described in Section 3.  For the purposes
of this study multiple sets of DRL experiments were conducted, using 4 random seeds (1-4)
for training and a single random seed (6) across all evaluations. The control input for the platoon
leader during training and evaluation is randomly sampled from a gaussian distribution
with a standard deviation of $0.1m/s^2$.

\section{Inter-FRL}
In order to evaluate the effectiveness of Inter-FRL relative to the base case where a
DRL model is trained using DDPG without FRL, 4 experiments are conducted without
Inter-FRL (no-FRL), and 8 with.  For each of the 12 conducted experiments, 2 platoons
with 2 vehicles were trained using one of the four random seeds.  Once training
across the four seeds has completed, the cumulative reward for a single evaluation
episode is evaluated.  For the experiments using Inter-FRL, two aggregation methods
are examined.  First, the gradients of each model are averaged during training, and
second, the model weights are averaged.  The multi platoon system trains and shares
the aggregated parameters (gradients or weights) from vehicle to vehicle with the same
index across platoons. The federated server is responsible for performing the averaging,
and each vehicle performs a training episode with the averaged parameters in addition
to their local training episodes in accordance with the FRL update delay and FRL
cutoff ratio (see Table \ref{tab:frlhyps}). Note that here-after Inter-FRL with gradient
aggregation is denoted Inter-FRLGA, and Inter-FRL with weight aggregation is denoted
Inter-FRLWA.


\subsection{Performance Across 4 Random Seeds}
The performance for each of the systems is calculated by averaging the cumulative reward
of each vehicle in the 2 vehicle 2 platoon system, as summarized in Table
\ref{tab:interfrl-summary}.  For each of the 3 cases (no-FRL, Inter-FRLGA and Inter-FRLWA),
training sessions were run using 4 random seeds.  In order to determine the highest
performing system overall, an average and standard deviation is obtained from the result
of training using the 4 random seeds. From Table \ref{tab:interfrl-summary}, it is
observed that both Inter-FRL scenarios using gradient and weight aggregation provide
large performance increases to that of the base case (no-FRL).

% Table generated by Excel2LaTeX from sheet 'changed'
\begin{table}[H]
  \centering
  \scriptsize
  \caption{Performance after training across 4 random seeds. Each simulation result contains 600 time steps.}
    \begin{tabular}{lrrlrrr} \toprule
    \textbf{Training Method} & \multicolumn{1}{l}{\textbf{Seed 1}} & \multicolumn{1}{l}{\textbf{Seed 2}} & \multicolumn{1}{l}{\textbf{Seed 3}} & \multicolumn{1}{l}{\textbf{Seed 4}} & \multicolumn{1}{l}{\textbf{Avg. System Reward}} & \multicolumn{1}{l}{\textbf{Std. Deviation}} \\ \midrule
    No-FRL & -3.73 & -2.89 & -4.69 & -3.38 & -3.67 & 0.66 \\
    Inter-FRLGA & -2.79 & -2.81 & -3.05 & -2.76 & -2.85 & 0.11 \\
    Inter-FRLWA & -2.64 & -2.88 & -2.92 & -2.93 & -2.84 & 0.12 \\ \bottomrule
    \end{tabular}%
  \label{tab:interfrl-summary}%
\end{table}%

 \subsection{Convergence Properties}
The cumulative reward is calculated over each training episode, and a moving average
is computed over 40 episodes to generate Figures
\ref{fig:avgep_pl1reward_nofrl}-\ref{fig:avgep_pl2reward_weight}.  It can be seen that
the cumulative reward for Inter-FRLWA not only converges more rapidly than both no-FRL
and Inter-FRLGA, but Inter-FRLWA also appears to have a more stable training session as
indicated by the lower magnitude of the shaded area (the standard deviation across the
four random seeds).

\begin{figure}[H]
\centering
    \begin{subfigure}{\interfrlRewWidth\textwidth}
        \raggedleft
        \includesvg[width=0.99\textwidth]{assets/interfrl/inter_avgep_pl1reward_nofrl.svg}
       \caption{No-FRL: Platoon 1}\label{fig:avgep_pl1reward_nofrl}
    \end{subfigure}\hspace{\interfrlRewSpace}
    \begin{subfigure}{\interfrlRewWidth\textwidth}
        \raggedleft
        \includesvg[width=0.99\textwidth]{assets/interfrl/inter_avgep_pl1reward_grads.svg}
       \caption{Inter-FRLGA: Platoon 1}\label{fig:avgep_pl1reward_grads}
    \end{subfigure}\hspace{\interfrlRewSpace}
    \begin{subfigure}{\interfrlRewWidth\textwidth}
        \raggedleft
        \includesvg[width=0.99\textwidth]{assets/interfrl/inter_avgep_pl1reward_weight.svg}
       \caption{Inter-FRLWA: Platoon 1}\label{fig:avgep_pl1reward_weight}
    \end{subfigure}
    \begin{subfigure}{\interfrlRewWidth\textwidth}
        \raggedleft
        \includesvg[width=0.99\textwidth]{assets/interfrl/inter_avgep_pl2reward_nofrl.svg}
       \caption{No-FRL: Platoon 2}\label{fig:avgep_pl2reward_nofrl}
    \end{subfigure}\hspace{\interfrlRewSpace}
    \begin{subfigure}{\interfrlRewWidth\textwidth}
        \raggedleft
        \includesvg[width=0.99\textwidth]{assets/interfrl/inter_avgep_pl2reward_grads.svg}
       \caption{Inter-FRLGA: Platoon 2}\label{fig:avgep_pl2reward_grads}
    \end{subfigure}\hspace{\interfrlRewSpace}
    \begin{subfigure}{\interfrlRewWidth\textwidth}
        \raggedleft
        \includesvg[width=0.99\textwidth]{assets/interfrl/inter_avgep_pl2reward_weight.svg}
       \caption{Inter-FRLWA: Platoon 2}\label{fig:avgep_pl2reward_weight}
    \end{subfigure}
\caption{Average performance across 4 random seeds for a 2 platoon 2 vehicle scenario trained without FRL (Figures \ref{fig:avgep_pl1reward_nofrl}, \ref{fig:avgep_pl2reward_nofrl}), with Inter-FRLGA (Figures \ref{fig:avgep_pl1reward_grads}, \ref{fig:avgep_pl2reward_grads}), and with Inter-FRLWA (Figures \ref{fig:avgep_pl1reward_weight}, \ref{fig:avgep_pl2reward_weight}). The shaded areas represent the standard deviation across the 4 seeds.}
\label{fig:inter_avep_rew}
\end{figure}

\subsection{Test Results for One Episode}
In Figures \ref{fig:simresINTERFRLWA_p1} and \ref{fig:simresINTERFRLWA_p2}, a simulation is
performed over a single training episode plotting the jerk, along with the control
input $u_{i,k}$, acceleration $a_{i,k}$, velocity error $e_{vi,k}$, and position
error $e_{pi,k}$ for each platoon.  There are 2 platoons in the Inter-FRL scenario,
and a simulation is provided for each platoon. The simulation environment is subject
to initial conditions of ($e_{pi} = 1.0\:m$, $e_{vi}=1.0\:m/s$, $a_i = 0.03\:m/s^2$).
It can be seen that each DDPG agent for both vehicles within the platoons quickly
responds to the platoon leader's control input $u_{i,k}$ to bring position error,
velocity error and control input to zero.  In addition, each DDPG agent closely
approximates the gaussian random input of the platoon leader, eliminating noise in
the response to maintain smooth tracking across the episode. Finally, each DDPG agent
in the platoon also minimizes the jerk effectively. These results are indicative of
both a good design of the reward function \eqref{eqn:rewardB}, and also a suitable
selection of parameters $a, b, c$ and $d$ in \eqref{eqn:rewardB}.

\begin{figure}[H]
\centering
    \begin{subfigure}{0.45\textwidth}
        \centering
        \includesvg[width=0.75\textwidth]{assets/interfrl/inter_sim_guassian_p1.svg}
       \caption{Platoon 1}\label{fig:simresINTERFRLWA_p1}
    \end{subfigure}\hspace{\interfrlRewSpace}
    \begin{subfigure}{0.45\textwidth}
        \centering
        \includesvg[width=0.75\textwidth]{assets/interfrl/inter_sim_guassian_p2.svg}
       \caption{Platoon 2}\label{fig:simresINTERFRLWA_p2}
    \end{subfigure}
\label{fig:interfrlsim}
\caption{Results for a specific 60s test episode using the 2 vehicle 2 platoon environment
trained using Inter-FRL with weight aggregation.}
\end{figure}

\section{Intra-FRL}
In order to evaluate the effectiveness of Intra-FRL relative to the base AV platooning
scenario, 4 experiments are conducted without Intra-FRL (no-FRL), and 8 with.  For each of
the conducted experiments, 1 platoon with 2 vehicles is trained using 4 random seeds.  A
single platoon is required for studying Intra-FRL as parameters are shared amongst
vehicles within the platoon (no sharing is performed from vehicle's in one platoon to
another).  Once training across the four seeds has completed, the cumulative reward for
a single evaluation episode is evaluated.  Similar to the experiments using Inter-FRL,
two aggregation methods are examined.  First, the gradients of each model are averaged
during training, and second, the model weights are averaged.  The platoon trains and
shares the aggregated parameters (gradients or weights) from vehicle to vehicle such
that data is averaged and updated amongst vehicles within the same platoon. The federated
server is responsible for performing the averaging, and each vehicle performs a training
episode with the averaged parameters in addition to their local training episodes in
accordance with the FRL update delay and FRL cutoff ratio (see Table \ref{tab:frlhyps}).
Note that here-after Intra-FRL with gradient aggregation is denoted Intra-FRLGA, and
Intra-FRL with weight aggregation is denoted Intra-FRLWA.


\subsection{Performance Across 4 Random Seeds}
The performance for the platoon is calculated by averaging the cumulative reward generated
by the simulation for each of the 4 random seeds and is summarized in Table
\ref{tab:intrafrl-summary}. The results in Table \ref{tab:intrafrl-summary} summarize
the performance for no-FRL, Intra-FRLGA, and lastly Intra-FRLWA.  It is observed that
Intra-FRLWA performs most favourably, followed by no-FRL and lastly Intra-FRLGA.


\begin{table}[H]
  \centering
  \scriptsize
  \caption{Performance after training across 4 random seeds. Each simulation result contains 600 time steps.}
    \begin{tabular}{lrrrrrr} \toprule
    \textbf{Training Method} & \multicolumn{1}{l}{\textbf{Seed 1}} & \multicolumn{1}{l}{\textbf{Seed 2}} & \multicolumn{1}{l}{\textbf{Seed 3}} & \multicolumn{1}{l}{\textbf{Seed 4}} & \multicolumn{1}{l}{\textbf{Avg. System Reward}} & \multicolumn{1}{l}{\textbf{Std. Deviation}} \\ \midrule
    No-FRL & -3.84 & -3.40 & -3.29 & -3.21 & -3.44 & 0.24 \\
    Intra-FRLGA & -2.85 & -8.05 & -4.23 & -2.99 & -4.53 & 2.10 \\
    Intra-FRLWA & -2.56 & -2.60 & -2.68 & -2.75 & -2.65 & 0.07 \\ \bottomrule
    \end{tabular}%
  \label{tab:intrafrl-summary}%
\end{table}%

 \subsection{Convergence Properties}

The cumulative reward is calculated over each training episode, and a moving average is
computed over 40 episodes to generate Figure \ref{fig:intra-convergence}.  Similar to the
Inter-FRL experiments, Intra-FRLWA shows the most favourable training results.  In addition,
the rate of convergence increases with Intra-FRLWA over no-FRL and Intra-FRLGA.  Lastly,
the stability during training is also shown to be improved as the standard deviation
across the four random seeds is much smaller than the other two cases (as evident in
the shaded regions of Figure \ref{fig:intra-convergence}).
\begin{figure}[H]
\centering
    \begin{subfigure}{\interfrlRewWidth\textwidth}
        \raggedleft
        \includesvg[width=\textwidth]{assets/intrafrl/intra_avgep_pl1reward_nofrl.svg}
       \caption{No-FRL}\label{fig:intra_avgep_pl1reward_nofrl}
    \end{subfigure}\hspace{\interfrlRewSpace}
    \begin{subfigure}{\interfrlRewWidth\textwidth}
        \raggedleft
        \includesvg[width=\textwidth]{assets/intrafrl/intra_avgep_pl1reward_grads.svg}
       \caption{Intra-FRLGA}\label{fig:intra_avgep_pl1reward_grads}
    \end{subfigure}\hspace{\interfrlRewSpace}
        \begin{subfigure}{\interfrlRewWidth\textwidth}
        \raggedleft
        \includesvg[width=\textwidth]{assets/intrafrl/intra_avgep_pl1reward_weight.svg}
       \caption{Intra-FRLWA}\label{fig:intra_avgep_pl1reward_weight}
    \end{subfigure}
\caption{Average performance across 4 random seeds for 1 platoon 2 vehicle scenario trained without FRL (Figure \ref{fig:avgep_pl1reward_nofrl}), Intra-FRLGA (Figure \ref{fig:avgep_pl1reward_grads}), and with Intra-FRLWA (Figure \ref{fig:avgep_pl1reward_weight}). The shaded areas represent the standard deviation across the 4 seeds.}
\label{fig:intra-convergence}
\end{figure}

\subsection{Test Results for One Episode}
A single simulation is performed on an episode plotting the jerk, along with the control
input $u_{i,k}$, acceleration $a_{i,k}$, velocity error $e_{vi,k}$, and position
error $e_{pi,k}$.  Figure \ref{fig:intraFRL-simresult} shows the precise control of
Intra-FRLWA on the environment.  The environment is initialized to the same conditions
as that of the Inter-FRLWA scenario
($e_{pi} = 1.0 m$, $e_{vi}=1.0 m/s$, $a_i = 0.03 m/s^2$), and each DDPG agent in the
platoon quickly and precisely tracks the gaussian random input $u_{i,k}$ from the
leader while minimizing position error $e_{pi,k}$, velocity error $e_{vi,k}$, control input $u_{i,k}$ and jerk.  Much
like the Inter-FRLWA scenario, it is observed that a strong optimization of the reward
function (\ref{eqn:rewardB}) has occured indicating a suitable selection of parameters $a,b,c$ and $d$ in the
reward function.

\begin{figure}[H]
    \centering
    \includesvg[width=0.35\textwidth]{assets/intrafrl/intra_sim_guassian_p1.svg}
    \caption{Results for a specific 60s test episode using the 2 vehicle 1 platoon
    environment trained using Intra-FRLWA.}
    \label{fig:intraFRL-simresult}
\end{figure}

\section{Comparison Between Inter and Intra-FRL}
The results for both Inter-FRL and Intra-FRL are summarized in Table \ref{tab:inter_vs_intra} below.

\begin{table}[H]
    \centering
    \scriptsize
    \caption{Performance after training across 4 random seeds for both Inter and Intra-FRL. Each simulation result contains 600 time steps.}
    \begin{tabular}{lrrlrrr} \toprule
    \textbf{Training Method} & \multicolumn{1}{l}{\textbf{Seed 1}} & \multicolumn{1}{l}{\textbf{Seed 2}} & \multicolumn{1}{l}{\textbf{Seed 3}} & \multicolumn{1}{l}{\textbf{Seed 4}} & \multicolumn{1}{l}{\textbf{Avg. System Reward}} & \multicolumn{1}{l}{\textbf{Std. Deviation}} \\ \midrule
        Inter-FRLGA & -2.79 & -2.81 & -3.05 & -2.76 & -2.85 & 0.11 \\
        Inter-FRLWA & -2.64 & -2.88 & -2.92 & -2.93 & -2.84 & 0.12 \\
        Intra-FRLGA & -2.85 & -8.05 & -4.23 & -2.99 & -4.53 & 2.10 \\
        Intra-FRLWA & -2.56 & -2.60 & -2.68 & -2.75 & -2.65 & 0.07 \\ \bottomrule
    \end{tabular}
    \label{tab:inter_vs_intra}
\end{table}

It is clear that using weight aggregation in both Inter-FRL and Intra-FRL is favourable
to gradient aggregation.  In addition, Intra-FRLWA provides the overall best result.
Intra-FRL likely converges to the best model due to conditions each agent experiences
during training.  For Inter-FRL, the environment is independent and identically
distributed. For Intra-FRL, each follower's training depends on the policy of the
preceding vehicle.  For the 2 vehicle scenario studied, vehicle 1 will converge prior
to vehicle 2 as vehicle 1 learns based on the stochastic random input generated by
the platoon leader. As vehicle 1 is training, vehicle 2 trains based off the policy
of vehicle 1. As previously stated, Inter-FRL parameters are shared amongst vehicles
in the same index across platoons, where-as Intra-FRL provides the advantage of
sharing parameters from preceding vehicles to following vehicles. Furthermore, the implementation
of Intra-FRL includes a directional parameter averaging.  For example, vehicle 1 does
not train with averaged parameters from the followers, but vehicle 2 has the advantage
of including vehicle 1's model in its averaging.  This directional averaging provides
an advantage to vehicle 2, as evidenced by the increased performance in Table
\ref{tab:inter_vs_intra}.

\section{Intra-FRL with Variant Number of Vehicles}
An additional factor to consider when evaluating FRL in relation to the no-FRL base
scenario is how FRL performs with increasing agents relative to no-FRL.  In this section,
12 experiments are conducted with no-FRL, and 12 with Intra-FRLWA. The set of 12
experiments for no-FRL and Intra-FRLWA are broken up by number of vehicle and random
seed.  The random seed is selected to be a value between 1 and 4, inclusive. In addition,
the platoons under study contain either 3, 4, or 5 vehicles. Once training has completed
for all experiments, the cumulative reward for each experiment is evaluated using a single
simulation episode in which the seed is kept constant.  Intra-FRLWA is used as the FRL
training strategy since Intra-FRLWA was identified to be the highest performing FRL
strategy in the previous section.

\subsection{Performance with Varying Number of Vehicles}
The performance for each experiment is calculated by taking the average cumulative episodic
reward across each vehicle in the platoon at the end of the simulation episode.  Table
\ref{tab:varypl} presents the results for no-FRL and Intra-FRLWA for platoons with 3, 4,
and 5 follower vehicles. Table \ref{tab:varypl} shows that Intra-FRLWA provides
favourable performance in all platoon lengths. A notable example of Intra-FRLWA's success
is highlighted when considering the poor performance of the 4 vehicle platoon trained with
no-FRL using seed 1. The Intra-FRLWA training strategy was able to overcome the performance
challenges, correcting the poor performance entirely.

% Table generated by Excel2LaTeX from sheet 'changed'
% Table generated by Excel2LaTeX from sheet '20220510200806295397'
\begin{table}[H]
  \centering
  \scriptsize
  \caption{Performance after training across 4 random seeds with varying platoon lengths. Each simulation result contains 600 time steps.}
    \begin{tabular}{lrrrrrrr} \toprule
    \textbf{Training Method} & \multicolumn{1}{l}{\textbf{No. Vehicles}} & \multicolumn{1}{l}{\textbf{Seed 1}} & \multicolumn{1}{l}{\textbf{Seed 2}} & \multicolumn{1}{l}{\textbf{Seed 3}} & \multicolumn{1}{l}{\textbf{Seed 4}} & \multicolumn{1}{l}{\textbf{Avg. System Reward}} & \multicolumn{1}{l}{\textbf{Std. Dev.}} \\ \midrule
    No-FRL & 3     & -3.64 & -3.28 & -3.76 & -3.52 & -3.55 & 0.20 \\
    No-FRL & 4     & -123.58 & -4.59 & -7.39 & -4.51 & -35.02 & 59.06 \\
    No-FRL & 5     & -4.90 & -5.94 & -6.76 & -6.11 & -5.93 & 0.77 \\
    Intra-FRLWA & 3     & -3.44 & -3.16 & -3.43 & -4.14 & -3.54 & 0.42 \\
    Intra-FRLWA & 4     & -3.67 & -3.56 & -4.10 & -3.60 & -3.73 & 0.25 \\
    Intra-FRLWA & 5     & -3.92 & -4.11 & -4.33 & -3.97 & -4.08 & 0.18 \\ \bottomrule
    \end{tabular}%
  \label{tab:varypl}%
\end{table}%

\subsection{Convergence Properties}
The cumulative reward is calculated over each training episode, and a moving average is
computed over 40 episodes to generate Figure \ref{fig:vehmult-convergence}.  Intra-FRLWA
shows favourable training performance to that of the no-FRL scenario for all platoon
lengths.  In addition, the rate of convergence is increased using Intra-FRLWA versus
no-FRL.  Furthermore, the shaded areas corresponding to standard deviation across the
seeds are reduced significantly indicating better stability across the seeds for
Intra-FRLWA than no-FRL.  Last, the overall stability is improved as shown by the large
noise reduction during training in Figures \ref{fig:reward_curve_p1_3v_intrafrlwa},
\ref{fig:reward_curve_p1_4v_intrafrlwa}, \ref{fig:reward_curve_p1_5v_intrafrlwa} when
compared with no-FRL's Figures \ref{fig:reward_curve_p1_3v_nofrl},
\ref{fig:reward_curve_p1_4v_nofrl}, \ref{fig:reward_curve_p1_5v_nofrl}.

\begin{figure}[H]
\centering
    \begin{subfigure}{\interfrlRewWidth\textwidth}
        \raggedleft
        \includesvg[width=0.99\textwidth]{assets/vehmult/vehmult_nofrl_3v_avgep_reward_pl1.svg}
      \caption{No-FRL: 3 Vehicles}\label{fig:reward_curve_p1_3v_nofrl}
    \end{subfigure}\hspace{\interfrlRewSpace}
    \begin{subfigure}{\interfrlRewWidth\textwidth}
        \raggedleft
        \includesvg[width=0.99\textwidth]{assets/vehmult/vehmult_nofrl_4v_avgep_reward_pl1.svg}
      \caption{No-FRL: 4 Vehicles}\label{fig:reward_curve_p1_4v_nofrl}
    \end{subfigure}
    \begin{subfigure}{\interfrlRewWidth\textwidth}
        \raggedleft
        \includesvg[width=0.99\textwidth]{assets/vehmult/vehmult_nofrl_5v_avgep_reward_pl1.svg}
      \caption{No-FRL: 5 Vehicles}\label{fig:reward_curve_p1_5v_nofrl}
    \end{subfigure}\hspace{\interfrlRewSpace}
    \begin{subfigure}{\interfrlRewWidth\textwidth}
        \raggedleft
        \includesvg[width=0.99\textwidth]{assets/vehmult/vehmult_intrafrlwa_3v_avgep_reward_pl1.svg}
      \caption{Intra-FRLWA: 3 Vehicles}\label{fig:reward_curve_p1_3v_intrafrlwa}
    \end{subfigure}\hspace{\interfrlRewSpace}
    \begin{subfigure}{\interfrlRewWidth\textwidth}
        \raggedleft
        \includesvg[width=0.99\textwidth]{assets/vehmult/vehmult_intrafrlwa_4v_avgep_reward_pl1.svg}
      \caption{Intra-FRLWA: 4 Vehicles}\label{fig:reward_curve_p1_4v_intrafrlwa}
    \end{subfigure}\hspace{\interfrlRewSpace}
    \begin{subfigure}{\interfrlRewWidth\textwidth}
        \raggedleft
        \includesvg[width=0.99\textwidth]{assets/vehmult/vehmult_intrafrlwa_5v_avgep_reward_pl1.svg}
      \caption{Intra-FRLWA: 5 Vehicles}\label{fig:reward_curve_p1_5v_intrafrlwa}
    \end{subfigure}
\caption{Average performance across 4 random seeds for 3 platoons with 3, 4 and 5 followers trained without FRL (Figures \ref{fig:reward_curve_p1_3v_nofrl}, \ref{fig:reward_curve_p1_4v_nofrl}, \ref{fig:reward_curve_p1_5v_nofrl}), and with Intra-FRLWA (Figures \ref{fig:reward_curve_p1_3v_intrafrlwa}, \ref{fig:reward_curve_p1_4v_intrafrlwa}, \ref{fig:reward_curve_p1_5v_intrafrlwa}). The shaded areas represent the standard deviation across the four seeds.}
\label{fig:vehmult-convergence}
\end{figure}

\subsection{Test Results for One Episode}
As with all previous sections, a single simulation is performed on a 60 second episode
plotting the jerk along with the control input $u_{i,k}$, acceleration $a_{i,k}$,
velocity error $e_{vi,k}$, and position error $e_{pi,k}$.
Figure \ref{fig:intraFRL-vehmult-simresult} showcases the ability of Intra-FRLWA to
control a 5 platoon environment precisely when compared to a platoon trained without
Intra-FRLWA.  The environment for Intra-FRLWA is initialized with the same values as
no-FRL, just like all previous experiments:
($e_{pi} = 1.0 m$, $e_{vi}=1.0 m/s$, $a_i = 0.03 m/s^2$). The Intra-FRLWA agent quickly
and precisely tracks the gaussian random control input $u_{i,k}$ from the leader minimizing
$e_{pi,k}$, $e_{vi,k}$, $u_{i,k}$ and jerk. In particular, the response for $e_{pi,k}$ and
$e_{vi,k}$ in the platoon trained using Intra-FRLWA
(Figure \ref{fig:sim_guassian_p1_5v_intrafrlwa}) appears to respond to the platoon
leader's input quicker and in a much smoother manner than that of the no-FRL
scenario (Figure \ref{fig:sim_guassian_p1_5v_nofrl}).

\begin{figure}[H]
\centering
    \begin{subfigure}{0.45\textwidth}
        \centering
        \includesvg[width=0.75\textwidth]{assets/vehmult/sim_guassian_p1_5v_nofrl.svg}
      \caption{No-FRL}\label{fig:sim_guassian_p1_5v_nofrl}
    \end{subfigure}\hspace{\interfrlRewSpace}
    \begin{subfigure}{0.45\textwidth}
        \centering
        \includesvg[width=0.75\textwidth]{assets/vehmult/sim_guassian_p1_5v_intrafrlwa.svg}
      \caption{Intra-FRLWA}\label{fig:sim_guassian_p1_5v_intrafrlwa}
    \end{subfigure}
\caption{Results for a specific 60s test episode using the 5 vehicle 1 platoon environment trained using no-FRL (Figure \ref{fig:sim_guassian_p1_5v_nofrl}), and with Intra-FRLWA (Figure \ref{fig:sim_guassian_p1_5v_intrafrlwa}).}
\label{fig:intraFRL-vehmult-simresult}
\end{figure}

The large difference in performance for no-FRL versus Intra-FRL can be explained by
understanding how Intra-FRLWA works. With no-FRL, each agent trains independently, and
the inputs to the following vehicles are directly outputted from the predecessors. Thus,
the followers farther back in the platoon take longer to train as their predecessors'
outputs can be highly variable while training. As the policies of the predecessors
converge, the policy of each follower can then begin to converge. This sequential
convergence from predecessor to follower can be seen in Figure \ref{fig:vehmult-convergence},
where the convergence during training is slower for vehicles 4 and 5 then it is for 3, 2
and 1. Intra-FRLWA helps to resolve this challenge by allowing vehicles to average
their model weights, thus distributing an aggregation of more mature predecessor parameters
amongst the platoon.