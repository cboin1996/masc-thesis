\chapter{Background} \label{chap:background}
In this chapter, high level background information is introduced on topics of autonomous
driving, FL, DRL, and lastly FRL

\section{Autonomous Driving}
In recent years there has been a surge in autonomous vehicle (AV) research, likely due 
to the technologies potential for increasing road safety, traffic throughput and fuel 
economy \cite{Makantasis2020a, Ye2019}. AV designs can vary in complexity from simple 
AV designs including hand designed rules with estimations for making driving possible, 
to complex systems composed of sensors and sophisticated software  \cite{Makantasis2020a}.  
The simplest designs lack the ability to generalize to the complex driving environments 
found in the real world \cite{Makantasis2020a}. In order to best model AV environments, 
two areas of research are often considered: supervised learning or RL approaches 
\cite{Makantasis2020a}.  Driving is considered to be a multi-agent interaction problem, 
and due to large variability of road data it can be quite challenging (or near impossible) 
to gather a dataset large enough to train a supervised model \cite{ElSallab2017a}.  
In addition, driving data is collected from humans, which can also limit an AI's ability 
to that of human level \cite{Ye2019}. In contrast, RL methods are known to generalize 
quite well \cite{Makantasis2020a} \cite{Makantasis2020a}, and furthermore DRL has seen 
large successes in autonomous driving and AV platooning research. 

In general there are two types of models for AV decision making: vehicle-following 
modeling or lane-changing modeling \cite{Ye2019}, the vehicle-following approach is 
selected as the AV model for this study.  In \cite{Zhu2018}, four typical AV platooning 
environments are presented: stimulus-based, safety-based, safety-distance, pyscho-physical 
and desired-measures. Stimulus-based environments reference their acceleration off 
of the relative distance and speed from the leading vehicle \cite{Zhu2018}. Safety 
distance models choose a gap large enough to stop at a specific velocity if the 
leading vehicle were to suddenly stop \cite{Zhu2018}.  Psycho-physical models 
operate with a driver varying their behavior with regard to traffic state through 
approaching the lead vehicle, steady state following, free driving, or braking \cite{Zhu2018}.  The desired measures model takes a select set of measures (for example speed or headway), and tries to minimize its actual measurements from the desired \cite{Zhu2018}. Vehicle-following models are based on following a vehicle in a single lane of a road with respect to a leading vehicle's actions \cite{Zhu2018}.  Vehicle-following models controlled using traditional controls methods are known to have: limited accuracy, poor generalization from a lack of data, and a lack of adaptive updating \cite{Zhu2018}. In order to address such limitations, AI is being explored as a method to control vehicle-following models \cite{Zhu2018}. In particular, DRL models have become a prominent area of research applied to AV platooning.

DRL approaches are model-free and the model may be inferred by the algorithm during 
training.  There are three elements important to DRL: training environment (ensuring 
variety in the training), choosing the proper algorithm for the case study (Q-learning 
and other algorithms are described below), and the creation of an effective reward 
function \cite{Ye2019}.  In order to achieve human like control, DRL has been accepted 
as one of the most promising approaches \cite{ElSallab2017a}.  DRL may be applied to 
human like driving models due to its ability to perform in complex situations 
\cite{Zhu2018}, mitigating common issues with vehicle following models \cite{Zhu2018}.  
DRL is promising for AV research for a few reasons: function approximation provided 
by deep neural networks (DNN)'s, ability to generalize from learning through training 
data (rather than parameter estimation as a result of data fitting) and continuous 
improvement \cite{Zhu2018}.

Several reinforcement learning models are useful to know when considering a DRL approach 
to AV: value-based RL, policy-based RL, DRL, Deep Q-Network (DQN) and Deep Deterministic 
Policy Gradient (DDPG). It is important to note that DDPG is the current prominent algorithm 
for applying DRL to AV driving scenarios, especially AV platooning. The following 
references are summarized in Chapter \ref{chap:litreview}, which are works that apply 
DDPG to AV platooning \cite{Lin2019, Song2020, Chu2019b, Peake2020}.  

\section{Federated Learning}
Google began the discussion of FL, showing FL's privacy advantages when compared 
to a centralized architecture \cite{Konecny2015, BrendanMcMahan2017a, McMahan2016FederatedLO}.  At the time of discovery, the core ideology of FL was to train machine learning models from distributed data sets across mobile devices while minimizing data leakage \cite{McMahan2016FederatedLO}.   Data transmitted during a FL process must be the minimal required to improve a model in order to reduce risk of data breaches \cite{BrendanMcMahan2017a}.  FL is defined in \cite{Yang2019a, IntelAI19} as a series of $N$ data owners ${\mathcal{F}_1,...\mathcal{F}_N}$ wishing to train a machine learning model by compiling data ${\mathcal{D}_1,...\mathcal{D}_N}$.  FL can be framed as an optimization problem containing the following characteristics \cite{IntelAI19}:
\begin{enumerate}
    \item Non-independent identical distributed (non-IID) datasets: Agents in a FL setting 
    may contain differing distributions amongst their data.
    \item Unbalanced number of data points: The sample space for each agent is variable.
    \item Large user base: Due to the large amounts of data required by deep learning 
    models, many participants with their own data sets are a likely solution to meet 
    the data demand for training a model.
    \item Communication challenges: FL systems rely on the users, introducing a variety 
    of user based challenges such as upload speed per user.
\end{enumerate}

FL strategies can be categorized based on the data's underlying characteristics 
\cite{Yang2019a, IntelAI19}.  

\begin{itemize}
    \item Horizontal FL (HFL): Applied to data sets with the same feature space (columns) 
    but different samples \cite{Yang2019a, IntelAI19}.  
    \item Vertical FL (VFL): Applied to data sets with the same sample space (rows in the 
    data set), but different features \cite{Yang2019a}.   
    \item Federated transfer learning (FTL): is essentially a combination of HFL and 
    VFL \cite{Yang2019a, IntelAI19}.  Thus, the feature and sample spaces of the data 
    sets are variable \cite{Yang2019a}.
\end{itemize}

In general, FL enables the online training of a single model with performance approaching 
that of the optimal centralized model \cite{Yang2019a}. A centralized model is the 
idealistic model for a distributed system, where a single model incorporates the intricacies 
of all agents into it's model structure directly \cite{Yang2019a}.  Often this centralized 
approach is desired for the control of a multi-agent system, however in the event of server 
failure all agents in the system would also fail \cite{Long2018}.  FL processes aim to 
avoid the risk of total system failure by producing a FL model from a training process 
that does not share any of the agents data points amongst themselves \cite{Yang2019a}.  
Ideally, the FL  model's performance should be close to that of the centralized model 
\cite{Yang2019a}.

As previously mentioned, \cite{BrendanMcMahan2017a} produced the FedAvg algorithm.  In 
FedAvg, each client takes a single step of gradient descent, computing the loss with 
local data. The Federated server then takes the weighted average of all resulting models 
\cite{BrendanMcMahan2017a}, returning the averaged parameters to each client for another 
local training step.  FedAvg provided an increase in performance for training high 
quality models, however privacy still needed to be considered further to ensure security 
during transmission \cite{BrendanMcMahan2017a}. \cite{Smith2017} employed a different 
approach from that of \cite{Yang2019a, BrendanMcMahan2017a} in the implementation of FL, 
by applying Multi-Task Learning (MTL) to the FL setting.  MTL allows for the training of 
separate models for each node in a system, but with the fitting of related models 
simultaneously \cite{Smith2017}. MTL labels any clustered, spare or low-rank structure 
between tasks in an a-priori fashion (without observation) -- task relationships are not 
known beforehand and may be inferred from the data structure itself \cite{Smith2017}.  
Although MTL has its benefits, FedAvg is widely used in literature and even used an an 
aggregation strategy within FRL. Because of FedAvg's relevance in literature, FedAvg is 
selected to be implemented on the AV platooning problem in this thesis. Lastly, it is 
important to understand that FL is only applied to supervised ML problems.  In order to 
apply FL to DRL problems, FRL has been developed. Before discussing FRL, first a brief 
introduction on the topics of RL and DRL is presented.

\section{Deep Reinforcement Learning}
RL is a learning from experience based optimization method useful for inferring actions as 
an output using an environment's states as input. In RL, an agent observes a given state, 
takes an action based on the state, and receives a reward as a metric to determine how 
`good' the action was.  As an RL agent explores an environment over a training episode, 
the agent aims to maximize the cumulative reward that is the sum of each reward received 
from taking an action at a time-step during a training episode.  The cumulative reward 
is maximized through learning an optimal state-action function known as the policy through 
a trial and error learning process.  DRL uses multi-layer neural networks in place of the 
policy function.  Some examples of well known DRL algorithms are: DQN, Advantage Actor 
Critic (A2C) and DDPG. Due to the success of DDPG in AV platooning research, the DDPG 
algorithm is selected for use in this thesis.  

Several RL and DRL algorithms are useful in building an understanding of DRL applications 
to AV platooning: value-based RL, policy-based RL, DQN and DDPG. In value-based RL a value 
function measures the quality of each state action pair predicting a long-term reward. The 
action value equation provides an expected reward given an action $u$ in state $x$, 
thereafter following the optimal policy. The action-value function is presented below 
from \cite{sutton2018reinforcement}

\begin{equation}
    q^{\pi}(x,u) = E[R_t|x_t = x, u_t = u]. \label{eqn:act_val}
\end{equation}

Q-learning aims to learn the action-value function from experience. The Q-function begins at 
random, and is updated according to the Bellman equation, presented below from  
\cite{sutton2018reinforcement}. 

\begin{equation}
    q(x,u) = E[r + \gamma\max_{u^{'}}(q(x', u'))] \label{eqn:bell}
\end{equation}

\noindent The Bellman equation assumes that the immediate reward $r$ plus the maximum 
future reward for the next state $x'$ produces the maximum future reward for state $x$ 
and action $u$ \cite{Zhu2018}.  Q-learning has become one of the most popular value-based 
methods, along with its variants: DQN, Double DQN, and Dueling DQN.\cite{sWang2017}.  

Policy based DRL also uses a value function $q(x, \theta)$ 
\cite{sutton2018reinforcement, sWang2017}. The policy, $\pi(u|x, \theta)$, is represented 
by neural networks, where $\theta$ represents the weights in a deep neural network (DNN) 
\cite{Zhu2018}.  A state space is given as input, and a continuous action space may be 
the output \cite{Zhu2018}. DRL uses a neural network to model the policy or value 
function \cite{Zhu2018}.  DQN is known to work in discrete action states, which has 
limited use in AV platoons where vehicles experience continuous action spaces 
\cite{Zhu2018, sWang2017}. The limitations of DQN in continuous environments led to 
the development of DDPG: a combination of DQN, actor-critic, and deterministic 
policy gradients, providing a DRL algorithm capable of training models with continuous 
action spaces \cite{Lillicrap2016, sWang2017}.  

Much like DQN, DDPG uses DNNs as the function approximators \cite{Zhu2018}.  Most 
optimization algorithms assume that samples of data are IID \cite{Lillicrap2016}.  
By the addition of a finite ``replay buffer'' which stores the state, action, reward 
and next state $(x, u, r, x^{'})$ for each state transition, minibatch sampling from 
the buffer allows the DDPG algorithm to learn across uncorrelated state transitions 
\cite{Lillicrap2016}.  In addition to the inclusion of a replay buffer, target 
networks are implemented to track the actor and critic weights slowly over time, stabilizing 
convergence during training \cite{Lillicrap2016, Zhu2018}.  The target network's are 
merely copies of the actor and critic networks, with weights updated slowly according to 
the hyper parameter $\tau$, such that 
$ \theta^{'} \leftarrow \tau\theta + (1-\tau)\theta^{'} $ \cite{Lillicrap2016}. 
Lastly, to address the challenge of exploration, an Ornstein-Uhlenbeck process is 
applied to sampled actions during training. Applying an Ornstein-Uhlenbeck process 
to continuous control problems with inertia (like that of the AV platooning problem) 
has been shown to improve exploration efficiency \cite{Lillicrap2016}.

The DDPG algorithm has been hugely successful in the control of continuous non 
linear systems: the inverted pendulum and robotic arm control are but a few examples 
of such successes \cite{Lim2020}. These successes in conjunction with the existing 
literature for AV platooning have been a large motivation for this thesis's application of 
DDPG to AV platooning. 

\section{Federated Reinforcement Learning}
Like FL, FRL aims to accelerate the training of DRL models through the sharing of parameters 
across a system. Most FRL research can be categorized into the following two categories: 
Horizontal Federated Reinforcement Learning (HFRL), and Vertical Federated Reinforcement 
Learning (VFRL).  The works in this Thesis are most closely related to HFRL.  HFRL and VFRL 
differ with respect to the structure of their environments and aggregation methods.  All 
agents in a HFRL architecture use isolated environments. It follows that each agent's action 
in a HFRL system has no effect on the other agents in the system.  HFRL architecture 
proposes the following training cycle for each agent: 

\begin{itemize}
    \item a training step is performed locally, 
    \item environment specific parameters are uploaded to the aggregation server, 
    known as the Federated Server,
    \item the Federated Server aggregates the parameters based on the aggregation 
    method (FedAvg is one example of an aggregation method),
    \item aggregated parameters are returned to each agent in the system for 
    another local training step.
\end{itemize} 

HFRL may be noted to have similarities to ``Parallel RL".  Parallel RL is a long studied 
field of RL, where agent gradients are transferred amongst each other 
\cite{Lim2020, Nadiger2019}.

On the contrary, VFRL defines an architecture composed 
of multiple agents exploring a single global environment. The agents in a VFRL system 
may observe partial or full states from the global model, and the actions of each agent 
have an affect the system as a whole. A VFRL architecture is composed of the following steps \cite{IntelAI19}:

\begin{itemize}
    \item All RL agents take an action according the the \textbf{same} environment.
    \item RL agents receive feedback from the environment (state, reward, etc.).
    \item RL agents compute 'mid products' - that is, the output from their corresponding model.  
    The output from their model is masked and sent to the Q-network agent.
    \item The Q-network encrypts each of the mid-products and trains itself using back-propagation. 
    The outputs from the Q-network are sent back to each agent \cite{IntelAI19}.
    \item All gradients received by each agent are encrypted and used to update their network.
\end{itemize}

Since decentralized models for AV platooning include a separate environment for each
vehicle in the platoon, we have chosen an HFRL architecture for our model.

\section{Summary}
Now that the theory and fundamentals of autonomous driving, FL, RL and lastly FRL 
have been introduced, an in depth summary of the state-of-the-art research in 
direct relation to the works of this thesis is presented in the next Chapter.