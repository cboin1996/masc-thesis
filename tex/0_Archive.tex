% THIS IS A PLACE FOR MISFIT INFORMATION THAT WAS REMOVED OR MAY BE USEFUL IN THE FUTURE.
\chapter{Literature Review}
\section{FRL}
An additional important note when applying FRL is the consideration of weighted averaging. 
The FedAvg \cite{BrendanMcMahan2017a} algorithm proposed weighted gradient averaging where the weight factor corresponded to the size of an agent training dataset.  The larger the dataset size was for the agent, the more weight it would vehiclery when performing averaging \cite{BrendanMcMahan2017a}.  Thus, applications of FRL often need to define a specific weighting factor to define which model in the system should be prioritized during aggregation.  A study by \cite{ZhangX2020} explores applying DRL to cellular vehicle to everything communication. \cite{ZhangX2020} describes a weighting factor dividing the sum batch size for all agents  by the training batch size for a specific agent.  In addition, \cite{LimHyun2021} explores how FRL using gradient aggregation can improve convergence speed and performance on the OpenAI-gym environments vehicletPole-V0, MountainvehicleContinuous-V0, Pendulum-V0 and Acrobot-V1. The works of \cite{LimHyun2021} assign weight to each agent's gradients with respect to the cumulative reward across the last 10 training episodes. By aggregating in this fashion, agents whose cumulative reward is more favourable achieves a stronger weight, and thus the gradients are favoured for the high performing agents across the system.  In addition, In addition, \cite{WangXiaofei2021} applies FRL to heterogenous edge caching.  \cite{WangXiaofei2021} uses the average of the following parameters for each agent over a number of episodes: average loss, training data size, episode number, batch size and hit rate (the network traffic). Lastly, \cite{Huang2021} applies FRL using model weight aggregation to Service Function Chains in network function virtualization enabled networks.  \cite{Huang2021} applies and evaluates the effectiveness of a reward based aggregation method like that of \cite{WangXiaofei2021} to a 'placement error', a parameter describing the ratio of error related to taking an action in the system.

\chapter{Literature Review} \label{chap:litreview}
In this chapter, an in-depth summary of the current state-of-the-art research related to FRL and DRL applied to AV platooning is presented. 

\section{Federated Reinforcement Learning}

Reinforcement learning is often a sequential learning process, and as such data is often non-IID with a small sample space \cite{sutton2018reinforcement}.  HFRL provides the ability to aggregate experience while increasing the sample efficiency thus providing more accurate and stable learning  \cite{IntelAI19}. It is important to note that few existing works apply FRL to autonomous driving at the time of this study.  Some of the current works studying HFRL to a variety of applications are summarized below.

% HFRL references here %
% \cite{Ren2019}
A study by \cite{Lim2020} aims to increase the performance of RL methods applied to multi-IoT device systems.  RL models trained on single devices are often unable to control devices in a similar albeit slightly different environment \cite{Lim2020}.  Currently, multiple devices need to be trained using separate RL agents \cite{Lim2020}. The methods proposed by \cite{Lim2020} sped up the learning process by 1.5 times for a two agent system. In a study by \cite{Nadiger2019}, the challenges in the personalization of dialogue managers, smart assistants and more are explored.  RL has proven to be successful in practice for personalized experiences; however long learning times and no sharing of data limits the ability for RL to be applied at scale.  Applying HFRL to atari non-playable characters in pong showed a median improvement of ~17\% for the personalization time \cite{Nadiger2019}. Lastly, \cite{Liu2019b} discusses RL as a promising algorithm for smart navigation systems, with the following challenges: long training times, poor generalization across environments, and storing data over long periods of time.  In order to address these problems \cite{Liu2019b} proposed the architecture `Lifelong FRL', which can be categorized as a HFRL problem.  It was found the Lifelong FRL increased the learning rate for the smart navigation system when tested on robots in a cloud robotic system \cite{Liu2019b}.  

An additional important note when applying FRL is the consideration of weighted averaging. 
The FedAvg \cite{BrendanMcMahan2017a} algorithm proposed weighted gradient averaging where the weight factor corresponded to the size of an agent training dataset.  The larger the dataset size was for the agent, the more weight it would vehiclery when performing averaging \cite{BrendanMcMahan2017a}.  Thus, applications of FRL often need to define a specific weighting factor to define which model in the system should be prioritized during aggregation.  A study by \cite{ZhangX2020} explores applying DRL to cellular vehicle to everything communication. \cite{ZhangX2020} describes a weighting factor dividing the sum batch size for all agents  by the training batch size for a specific agent.  In addition, \cite{LimHyun2021} explores how FRL using gradient aggregation can improve convergence speed and performance on the OpenAI-gym environments vehicletPole-V0, MountainvehicleContinuous-V0, Pendulum-V0 and Acrobot-V1. The works of \cite{LimHyun2021} assign weight to each agent's gradients with respect to the cumulative reward across the last 10 training episodes. By aggregating in this fashion, agents whose cumulative reward is more favourable achieves a stronger weight, and thus the gradients are favoured for the high performing agents across the system.  In addition, In addition, \cite{WangXiaofei2021} applies FRL to heterogenous edge caching.  \cite{WangXiaofei2021} uses the average of the following parameters for each agent over a number of episodes: average loss, training data size, episode number, batch size and hit rate (the network traffic). Lastly, \cite{Huang2021} applies FRL using model weight aggregation to Service Function Chains in network function virtualization enabled networks.  \cite{Huang2021} applies and evaluates the effectiveness of a reward based aggregation method like that of \cite{WangXiaofei2021} to a 'placement error', a parameter describing the ratio of error related to taking an action in the system.
The successes of the FedAvg algorithm as a means to improve performance and training times for systems has inspired further research into how aggregation methods should be applied.  The design of the aggregation method is crucial in providing performance benefits to that of the base case where FRL is not applied.  The FedAvg \cite{BrendanMcMahan2017a} algorithm proposed the averaging of gradients in the aggregation method.  In contrast, \cite{Liang2019} proposed using model weights in the aggregation method for AV steering control. Thus, FRL applications can differ based upon the selection of which parameter to use in the aggregation method.  A study by \cite{ZhangX2020} explores applying FRL to a decentralized DRL system optimizing cellular vehicle-to-everything communication. \cite{ZhangX2020} utilizes the model weights in the aggregation method, and describes a weighting factor dividing the sum batch size for all agents by the training batch size for a specific agent.  In addition, the works of \cite{LimHyun2021} explores how FRL using gradient aggregation can improve convergence speed and performance on the OpenAI-gym environments CartPole-V0, MountainvehicleContinuous-V0, Pendulum-V0 and Acrobot-V1.  \cite{LimHyun2021} determines that aggregating gradients using FRL creates high performing agents for each of the OpenAI-gym environments relative to models trained without FRL.  In addition, \cite{WangXiaofei2021} applies FRL to heterogeneous edge caching.  \cite{WangXiaofei2021} shows the effectiveness of FRL using weight aggregation to improve hit rate, reduce average delays in the network and offload traffic. Lastly, \cite{Huang2021} applies FRL using model weight aggregation to Service Function Chains in network function virtualization enabled networks. \cite{Huang2021} observes that FRL using model weight aggregation provides benefits to convergence speed, average reward and average resource consumption. Below is a summary of the previously mentioned FRL works categorized by their proposed aggregation method.

% Table generated by Excel2LaTeX from sheet 'Sheet1'
\begin{table}[H]
  \centering
  \caption{Summary of Existing Works in the Area of FRL Categorized by FRL Aggregation Method}
    \begin{tabular}{ll} \toprule
    \textbf{FRL Aggregation Strategies} & \textbf{References} \\ \midrule
    \midrule
    Gradient Averaging & \cite{BrendanMcMahan2017a}, \cite{LimHyun2021} \\
    Model Weight Averaging & \cite{Liang2019}, \cite{ZhangX2020}, \cite{WangXiaofei2021}, \cite{Huang2021} \\ \bottomrule
    \end{tabular}%
  \label{tab:addlabel}%
\end{table}%


Despite the differences in FRL applications within the aforementioned studies, each study maintain a similar goal: to improve the performance of each agent within the system.  None of the aforementioned works explore the differences in whether gradient or model weight aggregation is favourable in performance, and many of the works apply FRL to distributed network or communications environments.  It is the goal of this study to conclude whether model weight or gradient aggregation is favourable for AV platooning, as well as be one of the first (if not the first) to apply FRL to AV platooning.  

% Not revelant in this paper, perhaps the next it will be!
% A final important note to consider when implementing FRL is summarized from a study by \cite{Lim2020}:

% \begin{enumerate}
%     \item increasing the number of agents has a positive effect on FRL's performance,
%     \item transferring models which have completed training accelerates the training process for models in the early stages of training.
% \end{enumerate}

\section{Deep Reinforcement Learning applied to AV Platooning} \label{sec:avRL}

In order to improve the limitations of vehicle following models, DRL has been a steady area of research in the AV community with many authors contributing works to DRL applied to CACC \cite{Lin2019, Song2020, Chu2019b, Peake2020}.  In a study by \cite{Lin2019},  a DRL framework is designed to control a CACC AV platoon.  The DRL framework uses the Deep Deterministic Policy Gradient (DDPG) \cite{Lillicrap2016} algorithm and is found to have near-optimal performance \cite{Lin2019}.  In addition, \cite{Peake2020} identifies limitations in platooning with regard to the communication.  Through the application of a multi-agent reinforcement learning process, i.e. a policy gradient RL and LSTM network, the performance of a platoon containing 3-5 vehicles was improved upon that of current RL applications to platooning \cite{Peake2020}. Furthermore, Model Predictive Control (MPC) is the current state-of-the-art for real-time optimal control practices \cite{Lin_2021}.  The study performed by \cite{Lin_2021} applies both MPC and DRL methodologies to the AV platoon problem, observing a DRL model using the DDPG algorithm to have a 5.8\% episodic cost higher than the current state-of-the-art. The works of \cite{yan2021hybrid} propose a hybrid approach to the AV platooning problem where the platoon is modeled as a Markov Decision Process (MDP) in order to collect two rewards from the system at each time step simultaneously.  This approach also incorporates jerk, the rate of change of acceleration in the calculation of the reward for each vehicle in order to ensure passenger comfort \cite{yan2021hybrid}.  The hybrid strategy led to increased performance to that of the base DDPG algorithm, as the proposed framework switches between using classic CACC modeling and DDPG depending on the performance degradation of the DDPG algorithm \cite{yan2021hybrid}. In another study by \cite{Zhu2019}, a DRL model is formulated and trained using DDPG to be evaluated against real world driving data. Parameters such as time to collision, headway, and jerk were considered in the DRL model's reward function.  The DDPG algorithm provided favourable performance to that of the analysed human driving data \cite{Zhu2019}, with favourable time to collision results, more efficient driving via reduced vehicle headways, and improved passenger comfort with lower magnitudes of jerk.

When formulating the AV platooning problem as a DRL model DDPG is prominently selected as the algorithm for training.  DDPG's ability to handle continuous actions space and complex state's is perfect for the CACC platoon problem.  However, despite the DDPG algorithm's success in literature, there are still instability challenges related to the algorithm along with a time consuming hyper-parameter tuning process to account for the minute differences in vehicle models/dynamics amongst platoons. As previously discussed, FRL provides advantages in these areas where information sharing can accelerate performance during training and improve the performance of the system as a whole. In addition, the ability to share experience across like models has been proven to allow for fast convergence of models \cite{Lim2020}, which further optimizes the performance of DDPG when applied to AV platoons.

\chapter{Methodology}
%% WEIGHTS
As previously stated, the FRL server is designed to be flexible in how the aggregation is performed. If enabled, the FRL server is capable of performing a weighted average of either gradients or weights. For vehicle $v$ in platoon $P$ the weight factor $w_v$ is calculated as follows: given a current training episode $i$, a set of cumulative episodic rewards up to the current training episode $\mathcal{R}$, and a FRL weight window $\omega$:

\begin{equation}
   w_v = \dfrac{1}{\sum_{i=1}^{\omega} \mathcal{R}_{\omega-i}}
\end{equation}

Thus, the weighted parameters $p_w$ can be computed by the FRL server by applying the simple weighted average to each vehicle's parameters $p_v$ in the platoon of length $n$.

\begin{equation}
    p_w = \dfrac{\sum_{v=1}^{n} (w_{v} p_v)}{\sum_{v=1}^n w_{v}}
\end{equation}

\chapter{Results and Discussion}
\section{Experimental Setup}

In addition, (Weighted) denotes an aggregation method that implements weighted averaging. An additional hyperparmeter FRL weight window is defined for weighted aggregation to represent the number of episodes backward from the current training episode to use in the computation of the weighting factor.