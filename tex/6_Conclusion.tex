\chapter{Conclusions and Recommendations} \label{chap:conclusion}
\section{Conclusion}
In this thesis, we have formulated an AV platooning problem and successfully applied FRL in
a variety of methods to AV platooning. In addition, we proposed new approaches to applying
FRL to AV platoons: Inter-FRL and Intra-FRL.  By comparing FRL performance with both
gradient and weight averaging in the AV platooning scenario, it has been shown that
weight averaging was the optimal aggregation method regardless of using Inter-FRL or
Intra-FRL.  Furthermore, it was found that the Intra-FRLWA strategy was most advantageous
for applying FRL to AV platooning.  Finally, it was proven that applying Intra-FRLWA to
AV platoons up to 5 vehicles in length provided large performance advantages during and
after training when compared to AV platoons that were controlled without FRL. These
results are backed with simulations performed using models trained across four random
seeds, and an additional simulation set with variable platoon sizes.  The focus of this
thesis was on decentralized platoon control, where each follower in the platoon trains
locally with respect to their individual reward.

\section{Recommendations}
In the future, improvements to the system could be made implementing weighted averaging in
the FRL aggregation method. In addition, now that it is known that FRL is capable of out-performing
a decentralized system, training a centralized model for AV platoons
would allow for the performance of FRL to be analysed relative to the a best case scenario.
Moreover, in AV platooning communication delays can be
considered in the vehicle model to simulate the delays vehicles will experience through V2X
communications. Incorporating these changes will be helpful in progressing the works on FRL applied to AV
platoon control, as they are avenues for finding optimal performance for FRL. In addition,
the incorportation of communication delays is important to make the simulated AV platooning
environment closer to that of real world, where V2X communications can introduce lags in communication.
